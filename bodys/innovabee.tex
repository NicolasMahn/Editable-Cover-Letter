Mich begeistert Ihre Kombination aus grüner Wiese und Mittelstandsstabilität: Als SAP‑Gold‑Partner mit Du‑Kultur und direktem Draht zur Geschäftsführung schaffen Sie den Rahmen, KI wirksam zu gestalten. Ich möchte mit Ihnen definieren, wie Generative KI im SAP‑Kontext aussieht – nicht als Demo, sondern als messbare Prozessverbesserung. Als Early Adopter und Brückenbauer mache ich neue Tools verständlich – vom CEO bis zur Praktikantin.

Ich bringe einen Doppel‑M.Sc. in Business Application Architecture & Software Technology mit und arbeite an der Schnittstelle von LLM‑Agents, RAG und MLOps. Als Research Associate entwickelte ich eine experimentelle RAG‑Plattform samt Prompt‑Lehrmaterial; publikationsgestützt von ML für Geodaten (Springer) bis zu einem LLM‑Agentensystem für interaktive Datenanalyse (Masterarbeit). Praktika bei KPMG und der Deutschen Bahn schärften meinen Blick für Enterprise‑Governance. Über „MLOps & Kanban“ skaliere ich Lösungen mit klaren KPIs, Versionierung, Evaluation, Monitoring und CI/CD – und führe sie zuverlässig aus der PoC‑Falle in den Betrieb.

In den ersten 90 Tagen würde ich ein Use‑Case‑Portfolio (10–15 Ideen) erheben, priorisieren und 2–3 Low‑Risk‑High‑Impact‑Prototypen realisieren – z. B. Wissens‑RAG für Delivery/Pre‑Sales, Ticket‑Triage im Service oder einen Angebotsassistenten im Vertrieb. Parallel befähige ich das Team mit kompakten Prompt‑/Copilot‑Guidelines, Hands‑on‑Sessions und leichten Leitplanken. KPIs, Telemetrie/A‑B‑Tests und ein schlanker MLOps‑Pfad mit Rollback machen Wirkung sichtbar und skalierbar.
eoda verbindet genau das, was mich reizt: moderne KI-Technologien mit echten Anwendungsfällen bei großen Kunden – vom ersten Prototyp bis zum stabilen Betrieb. Eure Mission „Empowering Data-Driven Intelligence“ deckt sich mit meinem Ziel, verlässliche, nutzerzentrierte Lösungen in Produktion zu bringen. Besonders spannend finde ich eure Schwerpunkte KI-Agenten, RAG, MLOps, Dashboards und Trainings – hier kann ich fachlich wie methodisch anschließen und zugleich in einem Umfeld mit kurzen Entscheidungswegen Verantwortung übernehmen.

In meiner Masterarbeit entwickle ich ein LLM‑Agentensystem für interaktive Datenanalyse; zuvor habe ich einen RAG‑Prototyp samt Evaluations-Setups (Qualität, Relevanz, Kosten) aufgebaut und Lehrmaterialien zum Prompt Engineering erstellt. In Projekten habe ich End‑to‑End‑MLOps entlang CRISP‑ML(Q) gestaltet: klare Erfolgsmetriken, DVC, Tests und Containerisierung, CI/CD, Monitoring/Observability und inkrementelle Rollouts. Stationen bei der Deutschen Bahn (AI‑Infrastruktur) und KPMG (digitale Finanzen) geben mir das nötige Gespür für Enterprise‑Rahmenbedingungen. Publikationen zu Geodaten/Computer Vision (Springer) und VotingAid (CEUR) zeigen, dass ich Forschung pragmatisch in Wert überführe – inklusive Enablement und praxisnaher KI‑Governance.

Gern bringe ich dieses Profil ein: robuste Retrieval‑Designs, agentische Workflows mit Kosten-/Qualitätssteuerung, reproduzierbare Pipelines sowie verständliche Visualisierungen für Entscheider. Ich arbeite teamorientiert, lerne schnell und unterstütze Kolleg:innen und Kund:innen in Trainings. Deutsch und Englisch sind für mich sicher; standort- und zeitlich bin ich flexibel. Ich freue mich darauf, gemeinsam auszuloten, wo ich in KI‑Agenten bzw. RAG, MLOps oder Trainings den größten Hebel stifte.
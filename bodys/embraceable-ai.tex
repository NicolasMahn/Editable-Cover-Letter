Ihre „Conclusion Machines“ adressieren genau die Lücke zwischen KI‑Leistungsfähigkeit und Governance: nachvollziehbare Entscheidungen, Policy‑Kontext als First‑Class‑Objekt und volle Souveränität über Daten und Hardware. Genau dort liegt mein Schwerpunkt. Ich entwickle agentische Systeme mit policy‑grounded Reasoning, strukturierten Outputs, Quellenbezug und deterministischen Validierungsschritten. Sogar diesen Bewerbungsprozess automatisiere ich mit einem eigenen Agenten‑Workflow (RAG, Tool‑Use, Qualitäts‑ Checks, Versionierung) – als kleiner, transparenter Beleg meiner Arbeitsweise.

In meiner Masterarbeit baute ich ein LLM‑Agentensystem für interaktive Datenanalyse; zuvor eine experimentelle RAG‑Datenbank und Lehrmaterialien zu Prompt‑Engineering. Entlang CRISP‑ML(Q) setze ich End‑to‑End‑Pipelines mit Daten-/Modell‑Versionierung, CI/CD, Monitoring und klaren Qualitätsmetriken auf – um „Pilot‑Purgatory“ zu vermeiden. Praxisnähe bringe ich aus KPMG (Digital Finance) und Deutsche Bahn (AI‑Infrastruktur) mit; domänen- und governance‑nahe Use Cases sowie Veröffentlichungen (u. a. Springer 2023, CEUR 2025) untermauern mein methodisches Profil.

Gern skizziere ich kurzfristig einen Proof-of-Value: etwa einen Conclusion Worker, der Mietverträge analysiert, relevante IFRS-Vorschriften und kundenspezifische Richtlinien \\ berücksichtigt und daraus einen strukturierten Entscheidungs- und Quellenreport erstellt.
Dazu gehört ein OpenAI-kompatibles Evaluations-Setup mit Goldlabels, Regressionstests und klarer Fehlerklassifikation, ergänzt durch deterministische Validierungen und eine einfache Deployment-Skizze mit Monitoring und Governance-Anbindung.
Langfristig sehe ich Potenzial, gemeinsam Policy-Stores, Begründungsgraphen und wiederverwendbare Workflow-Blueprints für Back-Office, öffentlichen Sektor und Industrie aufzubauen – praxisnah, nachvollziehbar und skalierbar.
NEURA’s goal of building cognitive, human-centered robots resonates with my focus on perception, language-driven agents, and decision-making. I’m motivated by the practical challenge of grounding AI in the real world—making models not only accurate, but responsive, safe, and useful in human-machine interaction. The cross-functional setup at Metzingen/Riederich is exactly where I want to apply and deepen my skills.

I bring hands-on experience with end-to-end ML pipelines and reproducible experimentation. At MIR, I built a geospatial vision workflow for bike-path detection (data curation through evaluation), which resulted in a Springer publication and strengthened my grasp of detection/segmentation, metrics, and iteration under performance constraints. At KISS, I prototyped an experimental RAG system and am completing a thesis on an LLM agent for interactive data analysis (tool use, retrieval, grounding, evaluation)—experience that translates to multimodal vision-language interfaces. My B.Sc. thesis compared model-free RL algorithms; academically, I’ve focused on RL fundamentals with the intent to apply them to embodied tasks. I work primarily in Python (PyTorch, scikit-learn, Gym-style environments), collaborate closely across teams, and speak German (native) and English (C2).

I’d like to contribute to real-time perception for manipulation, multimodal fusion to improve interaction, and RL for adaptive control. I value clear baselines, aligned metrics, and careful ablations, and I’m eager to learn from NEURA’s experts while helping move promising ideas into products on-site in Metzingen.
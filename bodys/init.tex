INITs Leitstellensystem für den Echtzeit-Betrieb im ÖPNV verbindet genau die Themen, die mich motivieren: robuste Backend-Architektur, messbare Wirkung im Alltag und ein Beitrag zu nachhaltiger Mobilität. Die Entwicklung von Modulen für Fahrzeugortung oder Fahrplanüberwachung reizt mich besonders, weil hier Datenqualität, Latenz und Ausfallsicherheit unmittelbar über den Nutzen entscheiden. Ich möchte Lösungen bauen, die Dispatcherinnen und Disponenten spürbar entlasten – stabil, nachvollziehbar und gut wartbar.

Technisch passe ich zu Ihrem Stack: objektorientierte Java-Entwicklung mit Spring/JPA, Maven, REST, Docker sowie PostgreSQL gehören zu meinem Alltag; Microservice-Architekturen habe ich inklusive automatisierten Tests und CI/CD umgesetzt und in Scrum-Teams iterativ weiterentwickelt. Aus der Forschung bringe ich Geodaten- und Routing-Erfahrung mit (Springer-Publikation 2023); bei der Deutschen Bahn habe ich AI-Infrastruktur unter hohen Verfügbarkeitsanforderungen mitgestaltet. Ein experimentelles RAG-System inkl. sauberer Schnittstellen schärfte meinen Blick für Datenmodellierung und Performanz. In eventgetriebene Architekturen und Message Broker arbeite ich mich strukturiert ein und tausche mich gerne auf Deutsch wie Englisch fachlich aus.

Bei INIT möchte ich diese Stärken einbringen, um zuverlässige, gut getestete Backend-Module zu liefern, deren Mehrwert man täglich im ÖPNV sieht. Ich lerne schnell, teile Wissen gern und übernehme Verantwortung über den gesamten Entwicklungszyklus. Einstieg ist zeitnah möglich; ein Umzug nach Karlsruhe ist für mich gut vorstellbar.
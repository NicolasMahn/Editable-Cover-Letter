Ihre End-to-End-Ausrichtung – von Cloudplattform und Datenladestrecken bis zu BI-Reports und ML-Modellen – entspricht genau meiner Arbeitsweise. Ich schätze, dass ixto Innovation als Alltag lebt und Projekte konsequent an Business-KPIs ausrichtet. In diesem Umfeld möchte ich pragmatische, belastbare Lösungen mitgestalten: zwischen Kundenworkshop und Coding-Session, mit sauberer Engineering-Praxis und einem klaren Fokus auf Wirkung.

In meinem Doppel-Master habe ich entlang der gesamten Daten-Wertschöpfungskette gearbeitet. Für meine Masterarbeit entwickelte ich ein LLM-Agentensystem für interaktive Datenanalyse mit robustem Retrieval, Tool-Use und Evaluations-Setups; eigene RAG-Experimente vertieften Prompting-, Abfrage- und QS-Strategien. Im ML-Projekt zur digitalen Radwege-Erkennung (Springer-Publikation) verantwortete ich die modellseitige Pipeline inklusive belastbarer Metriken. Als Research Associate baute ich ein experimentelles RAG-Setup auf und erstellte didaktische Prompt-Engineering-Materialien. Technologisch arbeite ich end-to-end in Python, mit OpenAI/GPT-Stacks samt Evaluationspipelines, Databricks und Cloud-nahen CI/CD-, Container- und Versionierungsprozessen gemäß CRISP-ML(Q). Praktika bei KPMG und der Deutschen Bahn schärften mein Verständnis für Enterprise-Architekturen, Governance und Stakeholder-Management.

Gern bringe ich diese Erfahrung ein, um bei ixto Azure-/Databricks-Plattformen aufzubauen, LLM-Use-Cases (RAG, Agents, Evaluation) zu industrialisieren und Datenprodukte messbar in den Betrieb zu überführen. Microsoft Fabric und Databricks Mosaic AI vertiefe ich gezielt. Deutsche und englische Kommunikation, Teamgeist und Reisebereitschaft sind für mich selbstverständlich – mit der Motivation, gemeinsam transparente, entscheidungsrelevante Lösungen für Kunden in der DACH-Region zu schaffen.
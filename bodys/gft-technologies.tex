Die verantwortungsvolle, KI-zentrierte Ausrichtung von GFT und die Rolle als Brücke zwischen Sales und Delivery sprechen mich direkt an. Genau dort möchte ich komplexe GenAI-Technologien in belastbare, geschäftsrelevante Lösungsszenarien übersetzen und Thought Leadership mit messbarer Wertschöpfung verbinden – insbesondere in Financial Services und Industrie. GFTs Reichweite und Partnernetzwerk schaffen den Rahmen, um AI-Use-Cases nicht nur zu pilotieren, sondern skalierbar, sicher und kosteneffizient zu verankern.

Fachlich bringe ich tiefe LLM-/GenAI-Erfahrung mit: An der HFU habe ich eine experimentelle RAG-Umgebung aufgebaut und didaktische Materialien zu Prompt Engineering mit Guardrails, Evaluationsmetriken und Retrieval-Qualität entwickelt. Mit VotingAid (CEUR 2025) veröffentlichte ich einen agentenbasierten Entscheidungs-Use-Case; zuvor ML für Geodatenanalysen (Springer 2023). Praktika bei KPMG (Digital Finance) und der Deutschen Bahn (AI-Infrastruktur) schärften mein Verständnis für Enterprise-Architekturen, Data Pipelines, Compliance und MLOps-Grundlagen (Versionierung, Monitoring, Releasability). In Workshops, Use-Case-Designs und schnellen PoCs arbeite ich strukturiert, präzise und kollaborativ – auf Deutsch und Englisch.

Konkrete Go-to-Market-Ideen, die ich gern mit GFT vorantreibe: Trusted GenAI Research via RAG für Regulatorik und Compliance im Banking, LLM-gestützte Claims-Triage in der Versicherung sowie shopfloor-nahe Wissensassistenz und Qualitäts-/Instandhaltungs-Analytics in der Industrie – jeweils auditierbar, integrierbar in moderne IT-/Datenarchitekturen und mit klaren KPIs (Value, Risk, Feasibility). In den ersten 90 Tagen würde ich gemeinsam mit Sales, Tech Leads, Data Scientists und Marketing zwei bis drei Kernangebote schärfen, bei Schlüsselkunden pilotieren und auf Basis der Ergebnisse skalieren – und GFTs Thought Leadership durch Publikationen, Vorträge und Enablement-Materialien untermauern.

Dieses Schreiben entstand zum Großteil mit Unterstützung einer eigenen LLM-RAG-Pipeline – ein praktisches Beispiel dafür, wie ich GenAI bereits in Alltagsprozesse integriere.
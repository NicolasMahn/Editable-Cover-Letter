I’m excited by your vision to build a state‑of‑the‑art ML research environment for AI‑driven drug discovery—greenfield platforms, tight loops with drug design, and tooling that scales across accelerators. I work at the research–engineering boundary and enjoy turning prototypes into reliable, user‑centered systems. In a fast‑moving, Series A setting like yours, I can help design robust software and libraries that make experimentation faster, safer, and more reproducible.

Recently, I took an LLM‑agent for interactive data analysis from prototype to a stable, user‑facing tool: guardrails and tool‑use policies, comprehensive tests, and experiment/model/data versioning. I also built an experimental RAG database and evaluation harness to benchmark retrieval quality and latency, with prompt‑engineering materials adopted by researchers. Earlier, I led ML components of a geospatial analysis project published with Springer, collaborating closely with scientists. Technically, I’m Python‑first with PyTorch/TensorFlow, strong in data‑centric evaluation and error analysis, and comfortable with MLOps (Docker, CI/CD, reproducibility) and orchestration (Airflow/Kubeflow/Flyte/ZenML).

I’d bring this tooling mindset to your platform—creating infrastructure for large‑scale experimentation, optimizing efficiency and resilience, and iterating with domain experts from prototype to production. I’m eager to deepen JAX and multi‑accelerator training and to learn the biomedical context quickly to advance your drug‑discovery goals.
Your client’s mission to embed AI across products that drive efficiency, compliance, and smarter decisions resonates with me. In greenfield settings, I enjoy turning ambiguous needs into measurable systems; the blend of geoinformatics, data science, and enterprise software mirrors my path across LLM agents, RAG, and geospatial ML. The chance to set direction while working in small, trust-based teams is where I do my best work.

My current thesis builds an LLM-agent system for interactive data analysis with tool use, retrieval, and reliability at its core. At KISS, I implemented an experimental RAG datastore and developed prompt-engineering materials to improve knowledge access and explainability—directly relevant to documentation and policy-validation use cases. Earlier at MIR, I worked on ML for geospatial analysis; our Springer 2023 multi-stage bike-path study aligns with your geoinformatics footprint. I’m comfortable experimenting with pretrained models and crafting task-specific pipelines in Python and NLP, containerizing with Docker, and wiring lightweight CI/CD (GitHub Actions). I’ve presented on CRISP-ML(Q), evaluation, deployment, and monitoring, and my internships at KPMG and Deutsche Bahn taught me to align AI outputs with governance requirements and stakeholder expectations.

If selected, I’d prioritize a practical R&D roadmap: prototype RAG/LLM services for client documentation and policy validation, design evaluation harnesses and guardrails, and integrate features into product UIs/APIs to shorten knowledge-retrieval cycles. With fluent English and native German, I’m ready for direct customer interaction and to help establish reliable, scalable AI foundations across your international product landscape.
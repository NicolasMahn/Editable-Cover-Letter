I’m drawn to statworx’s holistic AI development approach—linking maturity assessments, model design, and scalable deployment—because it mirrors how I build solutions: reproducible, measurable, and tied to business impact. The focus on time-series use cases like forecasting, predictive maintenance, and anomaly detection aligns with my strengths and curiosity, and the consulting setup—flat teams, mentorship, and diverse industry contexts—matches how I learn and contribute best.

Technically, I bring solid Python skills and hands-on time-series experience across ARIMA/SARIMA, ETS, and Prophet, as well as ML-based forecasting with LightGBM/XGBoost. I’m rigorous about backtesting (rolling-origin), error metrics (MAE, RMSE, sMAPE), and feature engineering (lags/windows, calendars/holidays). On the MLOps side, I’ve produced materials on CRISP-ML(Q), CI/CD, versioning, and monitoring (schema checks, drift, baselines) and use Git, Docker, and MLflow/DVC to ensure reproducibility. I built a multi-stage ML pipeline for geospatial analytics published with Springer (2023) and developed an experimental RAG system plus MLOps learning content (CEUR 2025). Internships at KPMG and Deutsche Bahn taught me to translate between technical and business stakeholders; I communicate comfortably in German (native) and English (C1).

I’m keen to deepen modern methods like TFT, N-BEATS/N-HiTS, and time-series foundation models, and to expand my Azure/AWS/GCP skills. Projects in supply chain forecasting, sensor analytics, and multi-horizon forecasting particularly excite me, and I’m ready to travel within DACH to support clients and teams.
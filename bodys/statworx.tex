I’m drawn to statworx’s holistic AI development approach—linking maturity assessments, model design, and scalable deployment—because it mirrors how I build solutions: reproducible, measurable, and tied to business impact. The focus on time-series use cases like forecasting, predictive maintenance, and anomaly detection aligns with my strengths and curiosity, and the consulting setup—flat teams, mentorship, and diverse industry contexts—matches how I learn and contribute best.

Technically, I bring strong Python skills and an introductory understanding of time-series concepts such as ARIMA/SARIMA, ETS, and Prophet. I’ve experimented with ML-based forecasting using XGBoost and am eager to deepen my practical knowledge of statistical and hybrid approaches. I’m familiar with the principles of backtesting and error metrics like MAE and RMSE and am keen to apply these more systematically in real-world projects. On the MLOps side, I’ve explored frameworks such as CRISP-ML(Q) and Git/Docker-based reproducibility practices, and I’m expanding my skills in MLflow/DVC for model tracking and evaluation. I built a multi-stage ML pipeline for geospatial analytics published with Springer (2023) and developed an experimental RAG system with accompanying MLOps learning materials—initially within the KISS Research Project and later extended through my own publication, VotingAid. Internships at KPMG and Deutsche Bahn strengthened my ability to bridge technical and business perspectives, and I communicate confidently in both German (native) and English (C2 level).

I’m keen to deepen modern methods like TFT, N-BEATS/N-HiTS, and time-series foundation models, and to expand my Azure/AWS/GCP skills. Projects in supply chain forecasting, sensor analytics, and multi-horizon forecasting particularly excite me, and I’m ready to travel within DACH to support clients and teams.
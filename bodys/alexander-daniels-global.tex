Your client’s mission—placing a data‑efficient “expert operator” next to each machine—resonates with how I approach applied ML: augment human decision‑making with rigorous, small‑data models that are deployable in messy, real settings. As an early ML Research Engineer, I’d focus on pragmatic Bayesian Optimization for batch experiment suggestion, robust GP modeling under uncertainty, and knowledge transfer across machines, all backed by data‑quality systems that keep models trustworthy as conditions shift.

I’ve delivered under tight data constraints before: a multi‑stage geospatial CV pipeline with limited labels (Springer 2023) and an experimental RAG system to codify domain knowledge (CEUR 2025). These sharpened my statistical design, validation strategy, and pipeline reproducibility. In Python (NumPy/pandas, scikit‑learn, PyTorch/TensorFlow), I’m ready to implement BO/GP stacks with GPflow/BoTorch—e.g., constrained, batch, multi‑objective loops (qEI/qEHVI)—and wire in operator feedback and data‑quality gates. At Deutsche Bahn and KPMG, I worked closely with non‑ML stakeholders, translating constraints into iterative prototypes and production‑ready workflows.

While my industry tenure is adjacent to manufacturing, my core strengths—optimization under uncertainty, small‑data learning, and MLOps discipline—map directly to your customers’ needs. I’d be glad to produce a brief demo of a batch multi‑objective BO loop with data‑quality checks and operator‑feedback hooks within 1–2 weeks to illustrate fit and impact on efficiency and sustainability.
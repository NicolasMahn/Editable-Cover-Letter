Schwarz IT verbindet für mich genau die Hebel, die in einem Konzernumfeld Wirkung entfalten: komplexes Wissen und Abläufe aus Lidl, Kaufland, PreZero, Production und Digits mit GenAI so zugänglich zu machen, dass Fachbereiche schneller und sicherer entscheiden können – einfach, nachvollziehbar und nachhaltig. RAG und agentische Workflows sind dafür ideal: sie bringen interne Quellen und Tools zusammen, reduzieren Risiken und halten die Total Cost of Ownership im Blick.

Dazu bringe ich praxisnahe Erfahrung mit: Ein experimenteller RAG‑Stack in Python (Embeddings/Vektorsuche, Evaluationspipelines, Guardrails) sowie meine Masterarbeit zu LLM‑Agenten für interaktive Datenanalyse (Tool‑Use, Planung/Orchestrierung, strukturierte Outputs via Pydantic, Kosten-/Latenzoptimierung). Ich arbeite routiniert mit LangChain, sammle Erfahrung mit LangGraph und kenne PydanticAI. Produktionsnähe sichere ich über MLOps (CI/CD, Tests, Observability, Drift‑Monitoring, Alerting, Retraining), Containerisierung mit Docker und Grundlagen in Kubernetes; GCP/Vertex AI vertiefe ich gezielt. Datenseitig nutze ich u. a. PostgreSQL/pgvector und habe praktische Erfahrung mit MongoDB. In Projekten bei KPMG (Digital Finance) und der Deutschen Bahn (AI‑Infrastruktur) habe ich eng mit Plattform- und Fachteams gearbeitet und Ergebnisse publiziert (Springer 2023, CEUR 2025).

Konkrete Anknüpfungen sehe ich in: RAG‑Assistenten für SOPs/Policies mit Zitierlogik und Zugriffskontrollen, agentischer Ticket‑Triage samt ERP/CRM‑Tool‑Use sowie LLM‑gestützten Compliance‑Checks mit strukturierten Evidenzen. Gern starte ich mit einem schlanken Prototyp und klaren Metriken – und bringe Demos und erste Architekturvorschläge mit.
I’m applying speculatively for Munich because MaibornWolff’s project culture, cross‑industry clientele, and mentoring-first onboarding match how I like to build AI systems: collaborative, measurable, and production‑ready. Your blend of innovation and delivery—supported by flexible work and committed learning—creates the right setting to take clients from prototype to safe, cost‑aware AI in production.

My focus is LLM agents, Retrieval‑Augmented Generation, and applied ML. As a research associate, I built an experimental RAG corpus and created prompt‑engineering training; in another project I applied ML to geospatial data for bike‑path detection (Springer, 2023). My master’s thesis delivered an LLM agent system for interactive data analysis, and I recently published VotingAid (CEUR, 2025). Internships at KPMG and Deutsche Bahn grounded me in governance, cost/performance trade‑offs, and AI infrastructure. Practically, I architect robust RAG and agent pipelines, set up evaluation and observability (groundedness, hallucination rate, latency/cost KPIs), and lay MLOps foundations—containerization, CI/CD, monitoring, and cloud integration—guided by CRISP‑ML(Q) and DevOps. I also enjoy enabling teams through concise documentation and workshops.

I’m especially interested in agentic AI for industrial and mobility use cases, including digital twins as decision backbones. I’d welcome a conversation on where I could contribute most—as an AI/ML Engineer, LLM/RAG Engineer, Data Scientist, or AI Consultant—in full‑ or part‑time.
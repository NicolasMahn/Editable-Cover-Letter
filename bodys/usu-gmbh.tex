USU gestaltet die Servicewelt mit besserem Informationsfluss und effizienten Workflows – an dieser Schnittstelle aus Wissensmanagement, Datenpipelines und nutzerzentriertem Prozessdesign bringe ich praxisnahe KI-Expertise ein. Mit Double M.Sc. und Erfahrungen in Forschung und Industrie bewerbe ich mich initiativ, um Teams in Produktentwicklung, Research oder Consulting mit Fokus auf AI/LLM zu unterstützen. Ihr Mix aus Produktstärke, Beratungsnähe und Lernkultur (Buddyprogramm, U Step Up!) passt zu meinem analytisch-pragmatischen, kollaborativen Arbeiten.

Zuletzt habe ich eine experimentelle RAG-Datenbasis aufgebaut und Enablement-Materialien für Prompt Engineering entwickelt (KISS), eine mehrstufige geodatenbasierte ML-Pipeline bis zur Publikation umgesetzt (MIR, Springer) und bei KPMG/Deutsche Bahn an skalierten, compliance-festen Daten- und KI-Infrastrukturen mitgearbeitet. Darauf aufbauend würde ich bei USU robuste RAG-Workflows für Wissensbasen konzipieren (Halluzinationsreduktion, Evaluations-Harness), Service-Automatisierung im ITSM/CSM vorantreiben (Ticket-Klassifikation, Duplikaterkennung, Zusammenfassungen, Copilots) und die Operationalisierung sichern (MLOps, On-Prem/Hybrid, Governance). Zusätzlich setze ich KPI-Dashboards und A/B-Tests auf und befähige Teams durch Schulungen.

Ob als AI/LLM Engineer, Data Scientist/ML Engineer oder (Associate) Consultant: Ich möchte messbaren Nutzen für Ihre Kundschaft schnell validieren und nachhaltig in Produkte überführen. Rolle, Einstiegstermin und Arbeitsort stimme ich flexibel ab; Referenzprojekte stelle ich gerne bereit.
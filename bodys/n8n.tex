n8n’s mission to give teams the freedom of code with the speed of no‑code resonates with how I work: pragmatic, community‑minded, and automation‑first. To prove it, I automized my application process with an n8n + RAG workflow that ingests the job description, queries my vector store, drafts tailored materials, and posts status updates. I’m happy to share the workflow and a screenshot. That same mindset—build once, streamline, and feed learnings back—guides how I triage issues, close loops with contributors, and improve developer experience.

In my research roles, I created prompt‑engineering learning materials and supported peers by turning vague reports into minimal repros, adding tests, and documenting fixes. I’m fluent with Git/GitHub (issues, PRs, forks, Actions), modern JS/TS for integrations and tests, and debugging across external APIs/SDKs. If I joined, I’d start by standardizing labels and templates, then add a lightweight n8n workflow to route and dedupe new issues and request minimal repros automatically. I’d reproduce and fix small/medium bugs in community‑reported nodes, improve coverage for nodes with external dependencies to prevent regressions, and review PRs for API correctness, error handling, and node conventions—adding concise docs or tests where needed. My goal is simple: reduce maintenance load, elevate node quality, and help the community ship with confidence.
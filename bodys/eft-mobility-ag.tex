EFT Mobility’s mission to push battery and charger performance for autonomous flight resonates with me because it demands both disciplined execution and inventive, cross‑functional teamwork. I enjoy structuring fast, multi‑stakeholder development so that scope–time–cost, safety, and quality stay aligned. In a startup cadence with short cycles, I can bring order to ambiguity—turning requirements into clear plans, risks into tracked actions, and test evidence into decisions—while keeping engineers, suppliers, and customers informed in German and English.

In research roles and internships, I planned and delivered multi‑phase projects with milestones, backlogs, and review gates, coordinating across software, data, and infrastructure teams. My day‑to‑day includes Jira/Kanban, Git/GitLab CI, and Python analytics to turn validation data into dashboards for regression tracking and readiness reviews. I’m used to writing specifications, architecture sketches, acceptance criteria, and maintaining traceability for audits—useful for UN/CE/IEC‑compliant development. I bridge software, electrical, and mechanical work by translating firmware/API interfaces, telemetry, and diagnostics into actionable tasks, test plans, and acceptance checks. I am actively deepening BMS fundamentals (balancing, SoC/SoH, thermal) and power electronics and am prepared to support UN38.3, IEC 62133, and CE processes with a quality‑ and compliance‑oriented mindset.

I would quickly add value by owning planning, cross‑functional coordination, software/firmware integration, and validation/data flows in your battery and charger programs—managing change requests with structured impact assessments and maintaining a clear cadence of stakeholder updates. On‑site in Munich, I’m keen to learn from your team and help ship reliable, compliant units on schedule.
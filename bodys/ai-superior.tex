AI Superior’s focus on turning heterogeneous data into measurable business value—moving from rapid prototypes to production—aligns with how I like to work. I’m motivated by end-to-end problem solving and by a culture that pairs mentoring with high standards. The chance to contribute across visual, textual, sensor, and tabular projects while learning within a startup environment is exactly the context where I can be useful and grow.

In my recent roles, I built an experimental RAG database and authored prompt‑engineering materials at KISS, improving retrieval quality and evaluation design for LLM/NLP workflows. At MIR, I worked on geospatial ML for bike‑path detection (Springer 2023), emphasizing rigorous preprocessing, ground‑truth validation, and stakeholder‑friendly explanations. Internships at KPMG and Deutsche Bahn exposed me to enterprise constraints, compliance, and collaborative engineering. I’m comfortable with Python (pandas, NumPy, scikit‑learn, matplotlib/seaborn) and have hands‑on PyTorch experience. I’ve refactored notebooks into modular pipelines, added config/logging and experiment tracking, packaged code, containerized for consistency, and communicated results to mixed audiences—directly supporting your responsibilities around data processing, model development, validation, and internal tooling.

On day one, I can deliver clean data workflows, solid baselines, and disciplined evaluations, then iterate toward well‑tuned models and reproducible pipelines. I’m eager to deepen production‑grade MLOps (CI/CD, monitoring, cloud) and scale deep learning, while contributing reusable utilities and clear project summaries that help your teams move from “works” to “works reliably.”
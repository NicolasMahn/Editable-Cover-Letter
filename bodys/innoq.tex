INNOQ’s trust-first, discussion-friendly culture and focus on long-term, enablement-driven consulting resonate strongly with how I like to work. I value the autonomy to choose the right tools, the expectation to think critically, and the commitment to sharing knowledge through articles, talks, and trainings. That combination is ideal for helping clients adopt Data & AI responsibly while keeping solutions simple, maintainable, and aligned with business goals.

I bring hands-on experience building pragmatic GenAI and ML systems: RAG architectures and agent workflows, with evaluation and observability as first-class concerns and safety-by-design baked in. Recently, I developed an LLM-agent system for interactive data analysis and, as a research associate, built an experimental RAG corpus and created prompt-engineering learning materials. Earlier work includes ML for geospatial analysis (Springer, 2023) and a CEUR-published decision assistant (2025). I’m comfortable bridging data engineering to deployment (CI/CD, basic IaC) and approaching solutions through architecture reviews and lean delivery. Internships at KPMG and Deutsche Bahn exposed me to enterprise constraints and the importance of operability and compliance from day one.

At INNOQ, I would help clients design and evolve observable, secure, and cost-aware GenAI/ML platforms, while enabling their teams through workshops and clear documentation. I’m keen to contribute to your Data & AI, software architecture, and knowledge-transfer offerings—and to learn from your community’s collective experience.
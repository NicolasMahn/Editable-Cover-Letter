TRUMPF’s combination of product excellence and forward‑leaning digitalization resonates with me. As a double M.Sc. focused on LLM agents, RAG, and ML infrastructure, I’m motivated to design secure, scalable AI target architectures that work end‑to‑end across cloud, on‑premises, and edge—integrated with enterprise IT as well as product‑proximate systems. Your ambition to enable GenAI assistants, predictive maintenance, and visual inspection in a compliant, production‑grade manner aligns directly with my interests.

In my master’s thesis, I built an LLM‑agent system for interactive data analysis that blends retrieval‑augmented generation, tool calling, and evaluation workflows—patterns I would apply to assistant use cases with guardrails and observability. At the KISS Lab, I set up an experimental RAG corpus and authored prompt‑engineering materials with evaluation, safety, and governance guidelines—useful for policy‑based governance and clear architectural documentation. Earlier, at the MIR Lab, I led a multi‑stage computer‑vision pipeline for bike‑path detection (Springer 2023), strengthening my skills in robust data pipelines, validation, and explainability—transferable to visual inspection. Internships at Deutsche Bahn and KPMG added hands‑on experience with containerized, API‑first services in hybrid environments, including versioning, experiment tracking, model packaging, CI/CD, and operational monitoring.

I enjoy moderating architecture workstreams at the business–IT–AI interface and creating concise blueprints. In a case study on TRUMPF’s manufacturing and SYNCHRO, I saw strong parallels between Lean principles and resilient MLOps (flow, feedback, automation). I’m ready to evaluate platforms and model providers pragmatically, establish sustainable AI operations with privacy‑by‑design, and help deliver GenAI assistants, predictive maintenance, and visual inspection across cloud, on‑prem, and edge—while learning from and contributing to your interdisciplinary teams.
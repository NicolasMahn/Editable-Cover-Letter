Amadeus Fire überzeugt mich als verlässlicher Karrierepartner, der Entwickler passgenau in innovative IT‑Umgebungen bringt und dabei auf Qualität, Coaching und Flexibilität setzt. Genau in diesem Rahmen möchte ich meinen Einstieg als Softwareentwickler gestalten und Ihre Kunden in modernen Entwicklungsprojekten unterstützen – mit Fokus auf sauberer Umsetzung, transparenter Dokumentation und kontinuierlicher Verbesserung.

In meinem M.Sc.-Umfeld habe ich komplette Entwicklungszyklen durchlaufen: Entwicklung, Test, Debugging und Dokumentation. Konkret habe ich ein experimentelles RAG‑System (Indexierung, Retrieval, Evaluationspipeline) sowie ein modular aufgebautes LLM‑Agenten‑Framework für interaktive Datenanalyse umgesetzt. Darüber hinaus implementierte ich ML-/NLP‑Algorithmen in Geodatenprojekten (Springer‑Publikation 2023) und baute robuste, reproduzierbare Datenpipelines. Im Alltag arbeite ich mit Git, Docker, REST‑APIs, Jira und CI/CD‑Grundlagen; zuverlässige Unit-/Integrationstests und klare Architektur‑/Code‑Dokumentation sind für mich Standard. Aus Praktika bei KPMG und der Deutschen Bahn bringe ich Erfahrung in der Umsetzung technischer Spezifikationen, iterativen Verbesserungen und strukturierter Fehlersuche mit. Java beherrsche ich grundlegend, C++ ist vorhanden; beides erweitere ich bedarfsgerecht (z. B. Spring‑Stack).

Ich arbeite analytisch, teamorientiert und lerne schnell neue Domänen und Stacks. Gern identifiziere ich gemeinsam mit Ihnen eine passende Einsatzumgebung und liefere in Bonn, Frankfurt, Karlsruhe, Mannheim oder München – auch hybrid – messbaren Mehrwert für Ihre Kunden.
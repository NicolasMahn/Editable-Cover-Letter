Die Ausrichtung von mindcurv – Accenture Song, datengetriebene Transformation mit Snowflake, Databricks, Cloud sowie GenAI und ML voranzutreiben, entspricht genau meinem Profil. Ich möchte Kund:innen vom Geschäftsprozess her denken, tragfähige Datenarchitekturen konzipieren und sie bis zur Umsetzung von Pipelines, Security- und Cloud-Konzepten begleiten. Die Kombination aus früher Verantwortungsübernahme, Beratung auf Augenhöhe und einem dedizierten Lerntag pro Woche passt zu meiner Arbeitsweise.

In meiner aktuellen Tätigkeit als Research Associate habe ich ein experimentelles RAG‑System Ende‑zu‑Ende umgesetzt: Datenaufnahme/-aufbereitung, Vektorisierung und Evaluierung bis hin zu API‑Design und Analyse‑Frontends – inklusive Datenmodellierung und ETL/ELT‑Pipelines. Zuvor entwickelte ich in einem Geospatial‑Projekt eine ML‑Pipeline zur automatisierten Radweg‑Erkennung (Springer 2023). Praktika bei KPMG und der Deutschen Bahn vertieften datengetriebene Prozessanalyse, AI‑Infrastruktur sowie Security‑ und Governance‑Themen im Stakeholderdialog. Technisch bringe ich SQL, Python, Git, Docker und CI/CD sowie Erfahrung in Cloud‑/Container‑Umgebungen mit; Snowflake/Databricks kenne ich konzeptionell (Lakehouse/ELT, Skalierung, Governance) und baue dieses Wissen gezielt aus.

Ich arbeite strukturiert, lerne schnell und schätze Teamarbeit. Mit meinem BI‑Schwerpunkt im Doppel‑Master sowie verhandlungssicherem Deutsch und Englisch möchte ich dazu beitragen, bei Ihren Kund:innen belastbare Datenmodelle, aussagekräftige Dashboards und effiziente Datenpipelines zu schaffen – pragmatisch, transparent und wirkungsorientiert.